\documentclass{beamer}

\usepackage[portuguese]{babel}
\usepackage[utf8]{inputenc}
\usepackage{minted}

\title{AWK}
\author[João Marcelo Uchôa de Alencar]{João Marcelo Uchôa de Alencar}
\institute{Universidade Federal do Ceará - Quixadá}

\begin{document}
   \begin{frame}
      \titlepage
   \end{frame}

   \begin{frame}[fragile]
      \frametitle{O que é AWK?}
      \begin{itemize}
         \item O \textit{awk} é um comando que foi criado para englobar as capacidades do \textit{sed} e do \textit{grep}. \\
         \item A operação básica é pesquisar um conjunto de linhas de entrada, uma após a outra, procurando as que satisfaçam a um conjunto de \textbf{padrões ou condições especificadas} por você. \\
         \item Para cada padrão você pode especificar uma \textbf{ação}:\\
         \item $[$padrão$]$ $[$\{ ação \}$]$ \\
      \end{itemize}	 
      \begin{block}{Exemplo:}
      \begin{minted}{bash}
 awk '$1 == "Paula" {print $2, $3}' 
      \end{minted}
      \end{block}
      A instrução mostrada imprime o segundo e terceiro campos de cada linha entrada a partir do teclado cujo primeiro campo seja Paula. \\
      Veja a sutil diferença entre o uso de \$1, \$2 e \$3 no \textit{awk} e no \textit{shell}. \\
\end{frame}

   \begin{frame}[fragile]
      \frametitle{Uso do \textit{awk}}
      \begin{block}{Diretamente a partir do teclado:}
      \begin{minted}{bash}
awk '{<padrão e ação>}' [arquivo1] ... [arquivoN] 
      \end{minted}
      \end{block}
      \begin{block}{A partir de um arquivo de comando:}
      \begin{minted}{bash}
awk -f <arquivo de programa> <lista de arquivos> 
      \end{minted}
      \end{block}
      Por padrão, o separador é o TAB ou espaço em branco. \\
      A variável IFS não tem efeito no \textit{awk}. \\
      Os \textbf{padrões} se encaixam em três categorias:
      \begin{itemize}
         \item Expressões Relacionais (comparação)
	 \item Expressões Regulares
	 \item BEGIN e END
      \end{itemize}
\end{frame}

   \begin{frame}[fragile]
      \frametitle{Expressões Relacionais}
      \begin{table}
         \begin{tabular}{c|c}
	 \textbf{Operadores} & \textbf{Resultados} \\
	 \hline
         ==   & Igual a \\
	 $>$  & Maior que \\
	 $>=$ & Maior ou igual \\
	 $<$  & Menor que \\
	 $<=$ & Menor ou igual \\
         \&\& & e \\
	 $||$ & ou \\
	 !    & Não \\
         \hline 
         \end{tabular}
      \end{table}
      \small
      \begin{minted}{bash}
$ awk '$1 > "J" { print }' ../emailsordenados.txt
$ awk '$1 > "J" && $2 < "E" { print }' ../emailsordenados.txt
      \end{minted}
\end{frame}

   \begin{frame}[fragile]
      \frametitle{Expressões Regulares}
      Devemos colocar a expressão na parte do padrão, entre \slash\slash \\
      \small
      \begin{minted}{bash}
 $ awk '/^C/ { print }' ../emailsordenados.txt
 $ awk '$2 ~ /^C/ { print }' ../emailsordenados.txt
 $ awk '$1 !~ /^C/ { print }' ../emailsordenados.txt
 $ awk '$1 ~ /^[AB]/ { print }' ../emailsordenados.txt
 $ awk '$1 ~ /^(ANDRE|ANA)/ { print }' ../emailsordenados.txt
      \end{minted}
\end{frame}

   \begin{frame}[fragile]
      \frametitle{Padrões BEGIN e END}
      \begin{itemize}
         \item Padrão BEGIN: processamento anterior à entrada. Exemplo: cabeçalho. 
         \item Padrão END: processamento posterior à saída. Exemplo: totais e médias. 
      \end{itemize}
      \scriptsize
      \begin{minted}{bash}
$ awk 'BEGIN { print "Nome Completo" } { print }' ../emailsordenados.txt
$ awk 'END { print "Fim da Lista"} { print }' ../emailsordenados.txt
      \end{minted}
\end{frame}

   \begin{frame}
      \frametitle{Variáveis}
      \begin{table}
         \begin{tabular}{c|c}
         Variável & Significado \\
         \hline 
	 ARGC     &  Número de argumentos\\
	 ARGV     &  Vetor com parâmetros\\
	 FILENAME &  Nome do arquivo\\
	 FMR      &  Número do registro\\
	 FS       &  Separador \\
	 NF       &  Número de campos \\
	 NR       &  Quantidade de registros\\
	 OFMT     &  Formato para números\\
	 OFS      &  Separador na saída\\
	 ORS      &  Separador de registros\\
	 RS       &  Separador na entrada\\ 
         \end{tabular}
      \end{table}
   \end{frame}

   \begin{frame}[fragile]
      \frametitle{Variáveis}
      \footnotesize
      \begin{minted}{bash}
$ awk 'BEGIN { print "Nome do Alunos" }
> { print }
> END { print "Entradas=", NR }' ../emailsordenados.txt
awk '$1 ~ /^J/ { print; Soma=Soma+1 } 
> END { print Soma}' ../emailsordenados.txt
$ awk '
> BEGIN { printf "%15s %5s\n", "IP", "Latência" }
> { printf "%15s %5s\n", $1, $2, LatenciaTotal = LatenciaTotal + $2 }
> END { printf "\n Latência média: %5s\n", 
> LatenciaTotal / NR } ' ips_latencia.txt
      \end{minted}
\end{frame}

   \begin{frame}
      \frametitle{Operadores e Funções Matemáticas}

      \begin{table}
         \begin{tabular}{c|c}
         \textbf{Operadores} & \textbf{Significado} \\
	 \hline
         + & Soma \\
	 - & Diferença \\
	 * & Multiplicação \\
	 / & Divisão \\
	 \% & Resto da Divisão \\
	 ** & Exponenciação \\
	 = & Igualdade \\
	 \hline 
	 \textbf{Função} & \textbf{Significado} \\
	 \hline
          sqrt(x) & Raiz quadrada \\
	  sin (x) & Seno \\
	  log (x) & Logaritmo \\
	  rand (x) & Aleatório entre 0 e 1 \\
         \end{tabular}
      \end{table}
   \end{frame}


   \begin{frame}
      \frametitle{Funções de Cadeia de Caracteres}
      \begin{table}
         \begin{tabular}{c|c}
         \textbf{Função} & \textbf{Descrição} \\
         \hline
	 index(c1, c2) & Posição de c1 em c2 \\
	 length(c1) & Comprimento da cadeia \\ 
	 match(c1, exp) & Posição de c1 onde exp ocorre \\
	 split(c1, v [,c2]) & Divide c1 baseado em c2 \\
	 sprintf(fmt, lista) & Retorna lista formatada \\
	 sub(exp, c1 [, c2]) & Substitui exp em c2 por c1 \\
	 gsub(exp, c1 [, c2]) & Toda exp em c2 é substituída por c1 \\
	 substr(c1, p, n) & Retorna subcadeia em p de c1 com n caracteres \\
         \hline
         \end{tabular}
      \end{table}
   \end{frame}

   \begin{frame}[fragile]
      \frametitle{Funções de Cadeia de Caractere}
      \footnotesize
      \begin{minted}{bash}
$ awk 'BEGIN { print index("LINUX-UNIX", "X")}'
$ awk 'BEGIN { print length("LINUX-UNIX")}'
$ awk 'BEGIN { print match("LINUX-UNIX", /[UI]X/)}'
$ awk 'BEGIN { print match("LINUX-UNIX", /[UI]X$/)}'
$ awk 'BEGIN { OS="LINUX-UNIX" sub(/LINUX/, "BSD", OS) print OS}'
$ $ awk 'BEGIN {
> OS="LINUX-UNIX"
> sub (/LINUX/, "BSD", OS)
> print OS
}'
      \end{minted}
\end{frame}

   \begin{frame}
      \frametitle{Exemplo}
      \begin{itemize}
         \item Dado o arquivo \textit{emailsordenados.txt}, você conseguiria, usando o \textit{awk}, dizer quantos alunos tem o e-mail começando com a mesma letra do primeiro nome? E com a primeira letra do segundo nome? Considere maiúsculas e minúsculas iguais.
	 \item Lembrando a saída do comando \textit{ps aux}, faça agora um \textit{script} usando o \textit{awk} para contar a quantidade de memória virtual usada por determinado programa. Será que ficou mais fácil que usar \textit{grep} e \textit{sed}?
      \end{itemize}
   \end{frame}

   \begin{frame}[fragile]
      \frametitle{O comando \textit{if}}
      \begin{minted}{bash}
      if (expressao)
         {
	    comando1
	    comando2
	    ...
	    comandoN
	 }
      else 
         {
	    comando1
	    comando2
	    ...
	    comandoN
	 }
      \end{minted}
\end{frame}

   \begin{frame}[fragile]
      \frametitle{O comando \textit{while}}
      \begin{minted}{bash}
      while (expressao)
         {
	    comando1
	    comando2
	    ...
	    comandoN
	 }
      \end{minted}
\end{frame}

   \begin{frame}[fragile]
      \frametitle{O comando \textit{for}}
      \begin{minted}{bash}
      for (expressao1; expressao2; expressao3)
         {
	    comando1
	    comando2
	    ...
	    comandoN
	 }
      \end{minted}
\end{frame}

   \begin{frame}
      \frametitle{Saída dos Laços}
      \begin{itemize}
         \item \textbf{break}: desvia para primeira instrução após o laço
	 \item \textbf{continue}: avança para a primeira instrução da próxima interação do laço
	 \item \textbf{next}: pula para o próximo registro
	 \item \textbf{exit}: considera que o arquivo terminou, executa apenas ação END
      \end{itemize}
   \end{frame}

   \begin{frame}[fragile]
      \frametitle{O comando \textit{for}}
      \begin{minted}{bash}
$ awk '{ Registro [NR] = $0 } 
> END { print Registro [2] }' ../emailsordenados.txt 
      \end{minted}
      Os vetores também pode ser indexados por \textit{strings}.\\
\end{frame}

   \begin{frame}
      \frametitle{Saída com \textit{printf}}
      printf formato, expressao1, expressao2, ... , expressaoN \\
      \begin{table}
         \begin{tabular}{l|l}
	 Letra & Impressão \\
	 \hline
         c & Caractere \\
         d & Decimal \\
         e & Exponencial \\
         f & Ponto flutuante \\
         g & Ponto flutuante compacto \\
         o & Octal \\
         s & \textit{String} \\
         x & Hexadecimal \\
         \% & O símbolo \\
         \end{tabular}
      \end{table}
   \end{frame}

   \begin{frame}[fragile]
      \frametitle{Redirecionando a Saída}
      \footnotesize
      \begin{minted}{bash}
      $2 > 7.0 { print $1, $2 > "aprovados.txt" }
      $2 > 5.0 { print $1, $2 > ("/tmp/" ARQUIVO ) }

      {
         PrimeiraNota[$1] = $2
      }
      END {
         for (aluno in PrimeiraNota) 
	    print aluno, "\t" , PrimeiraNota[aluno] | "sort" 
      }

      \end{minted}
\end{frame}

   \begin{frame}[fragile]
      \frametitle{Acessando Parâmetros}
      \begin{minted}{bash}
      'BEGIN {
      for (i = 1; i < ARGC; i++)
         print ARGV [i], "\n"
      }' a b c
      \end{minted}
\end{frame}

\begin{frame}[fragile]
      \frametitle{Exemplo}
      Faça um \textit{script} marajas.awk que dado um arquivo de salários no formato abaixo imprima os professores que mais ganham por curso.
      \begin{minted}{bash}
      $ cat professores.txt
      NOME      CURSO      SALARIO
      Jeandro   Redes        10000
      Elvis     Engenharia   11000
      Hélder    Engenharia   15000
      João      Redes         1000
      Michel    Redes          800
      Jefferson Sistemas      5000
      Márcio    Sistemas      6000
      Marcos    Redes        11000
      $ awk -f marajas.awk professores.txt
      Engenharia: Elvis,  11000
      Redes:      Marcos, 11000
      Sistemas:   Márcio,  6000
      \end{minted}
\end{frame}

   \begin{frame}[fragile]
      \frametitle{Atividade}
      Coloque a resposta em \textit{atividades/atividade09}. Faça um \textit{script} chamado \textit{contaPalavras.awk} que ao ser invocado com o nome de um arquivo de texto diga quantas vezes cada palavra aparece no texto. \\
      \begin{minted}{bash}
      $ cat arquivo.txt
      a casa que vivo é boa.
      boa casa é.
      $ awk -f contaPalavras.awk arquivo.txt 
      Relatório de palavras.
      casa: 2
      boa:  2
      é:    2
      a:    1
      que:  1
      vivo: 1
      Total de Palavras analisadas: 9.
      \end{minted}
\end{frame}

   \begin{frame}[fragile]
      \frametitle{Atividade}
      Considere o seguinte arquivo \textit{notas.txt}
      \begin{minted}{bash}
Aluno:Nota1:Nota2:Nota3 
João Marcelo:10.0:10.0:10.0 
Alisson Barbosa:1.0:2.0:3.0
Jeandro Bezerra:7.0:8.0:9.0
Marcos Dantas:5.0:4.0:6.0
Michel Sales:3.0:4.0:2.0 
      \end{minted}
      Coloque na pasta \textit{atividades/atividade10} um \textit{script} chamado \textit{disciplina.awk} que realize a seguinte ação:
      \begin{minted}{bash}
$ awk -F: -f disciplina.awk notas.txt
Aluno:Situação:Média
João Marcelo:Aprovado:10.0
Alisson Barbosa:Reprovado:2.0
Jeandro Bezerra:Aprovado:8.0
Marcos Dantas:Final:5.0
Michel Sales:Reprovado:3.0
Média das Provas: 5.2 5.6 6.0
      \end{minted}
\end{frame}

\end{document}
