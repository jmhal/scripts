\documentclass{beamer}

\usepackage[portuguese]{babel}
\usepackage[utf8]{inputenc}
\usepackage{minted}

\title{Expressões Regulares}
\author[João Marcelo Uchôa de Alencar]{João Marcelo Uchôa de Alencar}
\institute{Universidade Federal do Ceará - Quixadá}

\begin{document}
   \begin{frame}
      \titlepage
   \end{frame}

   \begin{frame}
      \frametitle{O que é uma Expressão Regular?}
      \begin{itemize}
         \item Uma \textbf{Expressão Regular} (ER) é uma construção que utiliza pequenas ferramentas para obter uma determinada sequência de caracteres de um texto. 
         \item \textbf{Expressões Regulares} são escritas em uma linguagem formal que pode ser interpretada por um processador de expressões. 
      \end{itemize}
   \end{frame}

   \begin{frame}
      \frametitle{Para quem servem Expressões Regulares?}
      \begin{itemize}
         \item Procura e substituição de textos em editores e linguagens de programação.
         \item Validação de formatos de texto.
         \item Realce de sintaxe.
         \item Filtragem de informação. 
      \end{itemize}
   \end{frame}

   \begin{frame}
      \frametitle{Operações Básicas}
      \begin{block}{Alternância}
      Uma barra vertical ($|$) separa alternativas. A expressão 'casa$|$Casa' pode aceitar tanto a cadeia 'casa' quanto 'Casa'.
      \end{block}
      
      \begin{block}{Agrupamento}
      Parênteses () são usados para definir escopo e procedência. A expressão '(c$|$C)asa' é equivalente à anterior.
      \end{block}
      
      \begin{block}{Quantificação}
      Um quantificador após um \textit{token} ou agrupamento especifica a quantidade de vezes que o elemento precedente pode ocorrer. \\
      ? indica zero ou uma ocorrência. \\
      * indica zero ou mais ocorrências. \\
      + indica uma ou mais ocorrências. \\
      Chamamos ?,*,+ de metacaracteres.
      \end{block}
   \end{frame}

   \begin{frame}
      \frametitle{Metacaracteres}
      \begin{itemize}
         \item Âncoras
	 \item Representantes
	 \item Quantificadores
      \end{itemize}
   \end{frame}

   \begin{frame}
      \frametitle{Âncoras}
      \begin{table}
         \begin{tabular}{ c | l }
         \textbf{ER} & \textbf{Função}  \\
         \hline 
         \^{} & Pesquisar texto no início das linhas \\
         \hline
         \$ & Pesquisar texto no fim das linhas  \\
         \hline
         \textbackslash b & Pesquissar no início ou fim das palavras  \\
         \hline
         \textbackslash B & Negação de \textbackslash b  \\
         \hline
         \end{tabular}
      \end{table}
   \end{frame}

\begin{frame}[fragile]
   \frametitle{grep - Pesquisa arquivos por conteúdo}
   \begin{minted}{bash}
$ grep [-opções] [expressão] [arquivo1] [arquivo2] ...
   \end{minted}
   \begin{table}
      \begin{tabular}{ c | c }
         \textbf{-E} & \textbf{Ativa suporte estendido à expressões regulares} \\ 
	 \hline
	 -a & Força a tratar o arquivo como texto \\
	 \hline
         -c & Apenas informar quantas linhas contém a expressão \\
         \hline 
         -i & Não diferenciar maiúsculas e minúsculas \\
         \hline
         -l & Apenas informar qual dos arquivos contém a expressão \\
         \hline
         -v & Busca reversa \\
         \hline
         -n & Exibe o número da linha \\
         \hline
      \end{tabular}
   \end{table}
\end{frame}

   \begin{frame}
      \frametitle{Representantes}
      \begin{table}
         \begin{tabular}{ c | l | l}
         \textbf{ER} & \textbf{Função}  \\
         \hline 
         . & Ponto & Qualquer caractere uma vez \\
         \hline
         [] & Lista & Qualquer dos caracteres  \\
         \hline
         [\^{}] & Lista negada & Nenhum dos caracteres da lista  \\
         \hline
         \end{tabular}
      \end{table}
   \end{frame}

  \begin{frame}
      \frametitle{Quantificadores}
      \begin{table}
         \begin{tabular}{ c | l | l}
         \textbf{ER} & \textbf{Função}  \\
         \hline 
         ? & Opcional & Torna a entidade anterior opcional \\
         \hline
         * & Asterisco & Zero ou mais ocorrência  \\
         \hline
         + & Mais & Uma ou mais ocorrências  \\
         \hline
         \{\} & Chaves & Quantidade exata  \\
         \hline
         \end{tabular}
      \end{table}
   \end{frame}


   \begin{frame}
      \frametitle{Classes POSIX}
      \begin{table}
         \begin{tabular}{ c | l | l}
         [:alnum:] & Alfanuméricos & [A-Za-z0-9]   \\
         \hline 
         [:alpha:] & Alfabéticos & [A-Za-z] \\
         \hline
	 [:blank:] & Espaços  & [ \textbackslash t] \\
         \hline 
         [:cntrl:] & Controle & [\textbackslash x00-\textbackslash x1F\textbackslash x7F] \\
         \hline
	 [:digit:] & Dígitos &  [0-9] \\
         \hline 
         [:graph:] & Visíveis & [\textbackslash x21-\textbackslash x7E] \\
         \hline
	 [:lower:] & Minúsculas & [a-z] \\
         \hline 
         [:print:] & Visíveis e Espaços & [\textbackslash x21-\textbackslash x7E]  \\
         \hline
         [:punct:] & Pontuação & [....]  \\
         \hline 
         [:space:] & Espaços, mais nova linhas & [ \textbackslash t \textbackslash r \textbackslash n \textbackslash v \textbackslash f] \\
         \hline
	 [:upper:] & Maiúsculas & [A-Z] \\
         \hline 
         [:xdigit:] & Hexadecimais & [A-Fa-f0-9] \\
         \hline
         \end{tabular}
      \end{table}
   \end{frame}

   \begin{frame}
      \frametitle{Grupos}
      Os parênteses permitem criar grupos de caracteres que são avaliados como um só. \\ 	
      \begin{center}
         (vice-)?(governa$|$sena$|$verea$|$)dora?
      \end{center}
      Quantas palavras casam com essa expressão? \\
      O grupo no parênteses pode ser referenciado no resto da expressão.
      \begin{center}
         ([A-Za-z]+)\textbackslash 1
      \end{center}
      Toda e qualquer sequência de caracteres repetida.
   \end{frame}

   \begin{frame}
      \frametitle{Atividades - Parte I}
      Crie na sua pasta o diretório \textit{atividades/atividade03}. Copie para ele o arquivo \textit{/home/compartilhado/emailsordenados.txt}. Crie um arquivo chamado \textit{parte\_01.txt} e coloque nele, um por linha e na ordem abaixo, os seguintes comandos:
      \begin{enumerate}
         \item Um comando \textit{grep} que encontre todos os alunos cujo primeiro nome começa com A.
	 \item Um comando \textit{grep} que encontre todos os alunos que tem algum nome começando com A.
	 \item Um comando \textit{grep} que encontre todos os alunos cujo e-mails terminam com '.br'.
      \end{enumerate}
   \end{frame}

   \begin{frame}
      \frametitle{Atividades - Parte II}
      Copie o arquivo \textit{/home/compartilhado/expressoesregulares.txt} para a pasta \textit{atividades/atividade03}. Crie um arquivo \textit{parte\_02.txt} com os comandos \textit{grep}, um por linha, para as seguintes buscas:
      \begin{enumerate}
         \item Descubra quais linhas tem um espaço em branco antes da pontuação final.
	 \item Descubra quais linhas tem um sinal de pontuação seguido diretamente de uma letra, sem espaço entre ela e o sinal.
	 \item Descubra quais linhas não tem uma frase começando com letra maiúscula após a pontuação.
	 \item Descubra quais linhas não terminam com pontuação.
      \end{enumerate}
   \end{frame}

   \begin{frame}
      \frametitle{Atividades - Parte III}
      Crie o arquivo \textit{atividades/atividade03/parte\_03.txt} e coloque, uma por linha, as expressões regulares que representam as seguintes cadeias de caracteres:
      \begin{enumerate}
         \item A palavra "revista" no singular e no plural.
	 \item A palavra "letra" em qualquer combinação de letras maiúsculas ou minúsculas.
	 \item Os números inteiros.
	 \item Um número IP.
      \end{enumerate}
   \end{frame}

   \begin{frame}
      \frametitle{Atividades - Parte IV}
      Considerando o arquivo \textit{expressoesregulares.txt}, coloque em um arquivo \textit{parte\_04.txt} com um comando \textit{grep} para a seguinte busca:
      \begin{itemize}
         \item Descubra quais linhas contém palavras com mais que 8 caracteres.
      \end{itemize}
   \end{frame}

\end{document}

