\documentclass{beamer}

\usepackage[portuguese]{babel}
\usepackage[utf8]{inputenc}
\usepackage{minted}

\title{Resumo das Atividades}
\author[João Marcelo Uchôa de Alencar]{João Marcelo Uchôa de Alencar}
\institute{Universidade Federal do Ceará - Quixadá}

\begin{document}
   \begin{frame}
      \titlepage
   \end{frame}

   \begin{frame}
      \tableofcontents
   \end{frame}

\section{Atividade 1}
  \begin{frame}
     \begin{beamercolorbox}[sep=8pt,center,shadow=true,rounded=true]{title}
     \usebeamerfont{title}\insertsectionhead\par%
     \end{beamercolorbox}
  \end{frame}

   \begin{frame}
      \frametitle{Exercícios - Parte I}
      Este exercício deve ser feito na sua máquina local.
      \begin{itemize}
         \item Crie um diretório chamado \textit{ufc\_quixada}.
	 \item Crie um subdiretório chamado \textit{redes\_de\_computadores}.
	 \item Em \textit{redes\_de\_computadores}, crie um subdiretório chamado \textit{grade\_curricular}.
         \item Acesse \url{http://rc.quixada.ufc.br/matriz-curricular/} e anote o nome cada disciplina do curso.
         \item Para cada disciplina:
	 \begin{itemize}
	    \item Crie um subdiretório em \textit{ufc\_quixada/redes\_de\_computadores/grade\_curricular}.
	    \item O nome do diretório deve ser em minúsculas e os espaços devem ser trocados por \_.
	    \item Por exemplo: Internet e Arquitetura TCP/IP corresponde do diretório \textit{internet\_e\_arquitetura\_tcpip}
	 \end{itemize}
      \end{itemize}
   \end{frame}

   \begin{frame}
      \frametitle{Exercícios - Parte II}
      Este exercício é continuação do Exercício I.
      \begin{itemize}
         \item Acesse \url{www.quixada.ufc.br} e anote o nome e o primeiro sobrenome de cada professor que você já foi aluno.
         \item Crie um diretório \textit{ufc\_quixada/redes\_de\_computadores/professores}.
         \item Dentro do diretório criado, crie um diretório para cada professor. O nome do diretório de cada professor deve seguir o padrão \textit{nome\_sobrenome}. 
         \item Exemplo: deve existir o diretório \textit{ufc\_quixada/rede\_de\_computadores/professores/joao\_marcelo}.
      \end{itemize}
   \end{frame}

   \begin{frame}
      \frametitle{Exercícios - Parte III}
      Este exercício é continuação do Exercício II.
      \begin{itemize}
         \item Crie o diretório \textit{ufc\_quixada/redes\_de\_computadores/conquistas}.
	 \item Dentro de \textit{conquistas}, você vai criar um diretório para cada disciplina que já foi aprovado(a).
         \item Dentro do diretório criado, defina dois \textit{soft links}:
	 \begin{itemize}
	    \item \textit{programa}, que deve apontar para a pasta da disciplina em \textit{ufc\_quixada/redes\_de\_computadores/grade\_curricular}.
	    \item \textit{professor}, que deve apontar para a pasta do professor em \textit{ufc\_quixada/redes\_de\_computadores/professores}.
	 \end{itemize}
      \end{itemize}
   \end{frame}

   \begin{frame}
      \frametitle{Exercícios - Parte IV}
      
      \begin{itemize}
         \item Compacte sua pasta \textit{ufc\_quixada} no arquivo \textit{ufc\_quixada.tar.gz}.
	 \item No servidor, dentro do seu diretório de usuário, crie a pasta \textit{atividades}.
	 \item Crie o subdiretório \textit{atividade01}.
         \item Usando o comando \textit{scp}, copie o arquivo \textit{ufc\_quixada.tar.gz} para o diretório remoto \textit{atividades/atividade01}. 
         \item Descompacte o arquivo \textit{ufc\_quixada.tar.gz} e verifique se os \textit{links} estão corretos.
         \item Altere as permissões dos diretórios \textit{grade\_curricular} e \textit{professores} para apenas leitura.
      \end{itemize}
   \end{frame}
 
\section{Atividade 2}

   \begin{frame}
      \begin{beamercolorbox}[sep=8pt,center,shadow=true,rounded=true]{title}
      \usebeamerfont{title}\insertsectionhead\par%
      \end{beamercolorbox}
   \end{frame}

   \begin{frame}
      \frametitle{Exercícios - Parte I}
      Crie na sua pasta o diretório \textit{atividades/atividade02}. Copie para ele o arquivo \textit{/home/compartilhado/emailsordenados.txt}. Crie um arquivo chamado \textit{parte\_01.txt} e coloque nele, um por linha e na ordem abaixo, os seguintes comandos:
      \begin{enumerate}
         \item Um comando \textit{grep} que encontre todos os alunos cujo primeiro nome começa com A.
	 \item Um comando \textit{grep} que encontre todos os alunos que tem algum nome começando com A.
	 \item Um comando \textit{grep} que encontre todos os alunos cujo e-mails terminam com '.br'.
      \end{enumerate}
   \end{frame}

   \begin{frame}
      \frametitle{Exercícios - Parte II}
      Copie o arquivo \textit{/home/compartilhado/expressoesregulares.txt} para a pasta \textit{atividades/atividade02}. Crie um arquivo \textit{parte\_02.txt} com os comandos \textit{grep}, um por linha, para as seguintes buscas:
      \begin{enumerate}
         \item Descubra quais linhas tem um espaço em branco antes da pontuação final.
	 \item Descubra quais linhas tem um sinal de pontuação seguido diretamente de uma letra, sem espaço entre ela e o sinal.
	 \item Descubra quais linhas não tem uma frase começando com letra maiúscula após a pontuação.
	 \item Descubra quais linhas não terminam com pontuação
      \end{enumerate}
   \end{frame}

   \begin{frame}
      \frametitle{Exercícios - Parte III}
      Crie o arquivo \textit{atividades/atividade02/parte\_03.txt} e coloque, uma por linha, as expressões regulares que representam as seguintes cadeias de caracteres:
      \begin{enumerate}
         \item A palavra "revista" no singular e no plural
	 \item A palavra "letra" em qualquer combinação de letras maiúsculas ou minúsculas
	 \item Os números inteiros
	 \item Um número IP 
      \end{enumerate}
   \end{frame}

   \begin{frame}
      \frametitle{Exercícios - Parte IV}
      Considerando o arquivo \textit{expressoesregulares.txt}, coloque em um arquivo \textit{parte\_04.txt} com um comando \textit{grep} para a seguinte busca:
      \begin{itemize}
         \item Descubra quais linhas contém palavras com mais que 8 caracteres.
      \end{itemize}
   \end{frame}


\section{Atividade 3}
   \begin{frame}
      \begin{beamercolorbox}[sep=8pt,center,shadow=true,rounded=true]{title}
      \usebeamerfont{title}\insertsectionhead\par%
      \end{beamercolorbox}
   \end{frame}

   \begin{frame}
      \frametitle{Exercícios}
      Crie o diretório \textit{atividades/atividade03}. Copie o arquivo \textit{/home/compartilhando/auth.log} para o diretório criado. Crie um arquivo chamado \textit{auth\_analise.txt} com os seguintes comandos, um por linha:
      \begin{enumerate}
         \item Um comando que analise o arquivo \textit{auth.log} e informe quantas vezes tentaram fazer \textit{login} como \textit{root}.
	 \item Um comando que salve em um arquivo chamado \textit{malditos.txt} todas as tentativas de \textit{login} como \textit{root}.
	 \item Vasculhe o arquivo até encontrar um usuário que também seja aluno. Depois escreva o comando que conte quantas vezes ele ou ela tentou fazer o \textit{login} sem sucesso. 
      \end{enumerate}
      Abra o arquivo \textit{auth.log} em um editor de texto e estude seu formato antes de tentar definir os comandos.
   \end{frame}

\section{Atividade 4}
   \begin{frame}
      \begin{beamercolorbox}[sep=8pt,center,shadow=true,rounded=true]{title}
      \usebeamerfont{title}\insertsectionhead\par%
      \end{beamercolorbox}
   \end{frame}

   \begin{frame}
      \frametitle{Exercícios - Parte II}
      Crie o diretório \textit{atividades/atividade04}. Copie o arquivo \textit{/home/compartilhado/hosts/entrada.c} para o diretório criado. Crie o arquivo \textit{parte\_02.txt} e coloque os comandos sed que realizam as seguintes ações:
      \begin{enumerate}
         \item Troque todas as variáveis int para long e todas as float para double.
	 \item Adicione o arquivo de cabeçalho stdlib.h. 
      \end{enumerate}
   \end{frame}

   \begin{frame}
      \frametitle{Exercícios - Parte III}
      No diretório \textit{atividades/atividade04}, crie o arquivo \textit{parte\_03.txt}. Usando os comandos aprendidos, crie uma sequência de comandos ligados por $|$ 
que realize as ações abaixo. Coloque uma sequência por linha no arquivo \textit{parte\_03.txt}.
      \begin{itemize}
         \item Analisando $/$etc$/$passwd, liste, um por linha, os diferentes shells utilizados pelos usuários (último campo separado por :). 
	 \item Veja o arquivo $/$home$/$compartilhado$/$despesas.txt. Use os comandos para calcular o total de despesas.
      \end{itemize}
   \end{frame}

\section{Atividade 5}
   \begin{frame}
      \begin{beamercolorbox}[sep=8pt,center,shadow=true,rounded=true]{title}
      \usebeamerfont{title}\insertsectionhead\par%
      \end{beamercolorbox}
   \end{frame}
   \begin{frame}
      \frametitle{Exercícios Parte I}
      Coloque os \textit{scripts} criados em \textit{atividades/atividades05}:
      \begin{itemize}
         \item Escreva um script chamado \textbf{addN.sh} que recebe uma lista de números e retorna a soma deles.
	 \item Escreva um script chamado \textbf{menorMediaMaior.sh} que recebe uma lista de números e imprime na tela o menor deles, a média aritmética e o maior em outra linha. 
      \end{itemize}
   \end{frame}

\section{Atividade 6}
   \begin{frame}
      \begin{beamercolorbox}[sep=8pt,center,shadow=true,rounded=true]{title}
      \usebeamerfont{title}\insertsectionhead\par%
      \end{beamercolorbox}
   \end{frame}

   \begin{frame}
      \frametitle{Exercícios - Parte II}
      \scriptsize
      Coloque os \textit{scripts} e arquivos criados em \textit{atividades/atividades06}. 
      Considere um arquivo \textit{emails\_database.txt} que armazena usuários e e-mails com o seguinte formato para a linha: 
      nome completo:email \\ 
      Escreva os seguintes \textit{scripts}:
      \begin{itemize}
         \item Script \textbf{addUser.sh}: recebe como parâmetro o nome completo e email e os adiciona no arquivo. Se o arquivo não existir, deve ser criado.
	 \item Script \textbf{remUser.sh}: recebe como parâmetro o e-mail e remove a linha correspondente do arquivo.
	 \item Script \textbf{showUser.sh}: recebe como parâmetro o e-mail e exibe o nome do usuário.
	 \item Script \textbf{showDomains.sh}: recebe como parâmetro um domínio (gmail, yahoo, hotmail, ufc, etc) e retorna quantos usuários tem e-mail nesse domínio (ou semelhante) e exibe esses usuários.
	 \item Script \textbf{showFile.sh}: exibe o conteúdo do arquivo.
      \end{itemize}
   \end{frame}

\section{Atividade 7}
   \begin{frame}
      \begin{beamercolorbox}[sep=8pt,center,shadow=true,rounded=true]{title}
      \usebeamerfont{title}\insertsectionhead\par%
      \end{beamercolorbox}
   \end{frame}
  
   \begin{frame}
      \frametitle{Exercícios - Parte III}
      Coloque os \textit{scripts}, arquivos e pastas criados em \textit{atividades/atividades07}. Considere todos os arquivos do diretório /home/compartilhado/maillog: 
      \begin{itemize}
         \item Os arquivos não estão organizados por dia. Por exemplo mail.log.1 tem informação do dia 05\slash09 até o dia 11\slash09. Crie novos arquivos no formato mail\_DD\_MM.log de forma que cada arquivo contenha apenas as entradas do dia DD do mês MM. É necessário que os arquivos de entrada sejam fornecidos através de parâmetro.
	 \item Use o \textbf{xargs} para executar um comando que transfira os arquivos do 09 para o diretório 09, os arquivos do mês 08 para o diretório, etc. Por enquanto são apenas dois meses, mas faça seu comando para suportar quantidade maior.
      \end{itemize}
   \end{frame}

\section{Atividade 8}
   \begin{frame}
      \begin{beamercolorbox}[sep=8pt,center,shadow=true,rounded=true]{title}
      \usebeamerfont{title}\insertsectionhead\par%
      \end{beamercolorbox}
   \end{frame}

   \begin{frame}
      \frametitle{Exercício - Parte I}
      Escreva na pasta \textit{atividades/atividade08} um \textit{script} chamado \textbf{numeroOuNao.sh} que receba um parâmetro e afirme se o mesmo é um número ou não.
   \end{frame}

   \begin{frame}
      \frametitle{Exercício - Parte II}
      Na mesma pasta \textit{atividades/atividade08}, escreva um \textit{script} chamado \textit{param.sh} que verifique se o usuário passou parâmetros ou não. Caso não tenha passado parâmetros, deve imprimir apenas o nome do script. Caso tenha passado parâmetros, deve imprimir todos.
   \end{frame}

   \begin{frame}
      \frametitle{Exercício - Parte III}
      Na mesma pasta \textit{atividades/atividade08}, faça um \textit{script} chamado \textit{isfile.sh} que receba um parâmetro e verifique se é o nome de um arquivo ou diretório e informe se você tem permissão de escrita e leitura. \\
      Por exemplo: \\
      \$ .\slash testFile.sh /etc/hosts \\
      É um arquivo. \\
      Tem permissão de leitura. \\
      Não tem permissão de escrita. \\
   \end{frame}

   \begin{frame}
      \frametitle{Exercício - Parte VI}
      De volta ao diretório \textit{atividades/atividade08}, escreva um \textit{script} chamado \textbf{servico.sh}. Esse script deve receber um parâmetro.
      \begin{itemize}
         \item Se o parâmetro for \textit{start}, deve imprimir \textquotedblleft Iniciando Serviço\textquotedblright.
         \item Se o parâmetro for \textit{stop}, deve imprimir \textquotedblleft Parando Serviço\textquotedblright.
         \item Se o parâmetro for \textit{restart}, deve imprimir \textquotedblleft Reiniciando Serviço\textquotedblright.
         \item Se o parâmetro for qualquer outra coisa,  deve imprimir \textquotedblleft Uso: servico.sh (start$|$stop$|$restart) \textquotedblright.
      \end{itemize}
   \end{frame}

\section{Atividade 9}
   \begin{frame}
      \begin{beamercolorbox}[sep=8pt,center,shadow=true,rounded=true]{title}
      \usebeamerfont{title}\insertsectionhead\par%
      \end{beamercolorbox}
   \end{frame}

   \begin{frame}
      \frametitle{Exercícios I}
      Agora é com vocês !!!
      \begin{itemize}
         \item Crie o diretório \textit{atividades/atividade09}
	 \item Você vai desenvolver o \textit{script} \textit{latencia.sh}
	 \item Esse \textit{script} vai receber como parâmetro o nome de um arquivo de texto, contendo um endereço IP por linha.
	 \item O \textit{script} deve usar o comando \textbf{ping} para enviar dez pacotes ICMP, calculando o valor médio do tempo de resposta.
	 \item O \textit{script} deve imprimir uma lista de IP ordenada do menor para o maior tempo médio de resposta, informando além do endereço, o tempo de resposta médio.
      \end{itemize}
   \end{frame}

\section{Atividade 10}
   \begin{frame}
      \begin{beamercolorbox}[sep=8pt,center,shadow=true,rounded=true]{title}
      \usebeamerfont{title}\insertsectionhead\par%
      \end{beamercolorbox}
   \end{frame}

   \begin{frame}
      \frametitle{Exercícios - Parte I}
      Crie o diretório \textit{atividades/atividade10}. \\
      Escreva um \textit{script} chamado \textit{alertaDiretorio.sh} que verifica de minuto em minuto a quantidade de arquivos em um diretório. \\
      Caso a quantidade de arquivos se altere entre duas verificações, o \textit{script} deve enviar uma mensagem para o seu endereço de e-mail. \\
      Faça com que o \textit{script} fique em execução mesmo sem que seu usuário esteja logado. \\
   \end{frame}

\section{Atividade 11}
   \begin{frame}
      \begin{beamercolorbox}[sep=8pt,center,shadow=true,rounded=true]{title}
      \usebeamerfont{title}\insertsectionhead\par%
      \end{beamercolorbox}
   \end{frame}

   \begin{frame}
      \frametitle{Exercícios - Parte I}
      Coloque o \textit{script} resultante em \textit{atividades/atividades11}. \\
      Observe o arquivo \textit{/home/compartilhado/calculadora.sh}. Altere o programa para também aceitar operações em ponto flutuante com a vírgula como separador das casas decimais.
   \end{frame}

\section{Atividade 12}
    \begin{frame}
      \begin{beamercolorbox}[sep=8pt,center,shadow=true,rounded=true]{title}
      \usebeamerfont{title}\insertsectionhead\par%
      \end{beamercolorbox}
   \end{frame}
 
   \begin{frame}[fragile]
      \frametitle{Exercícios - Parte I}
      Faça um \textit{script} chamado \textit{contaPalavras.sh} na pasta \textit{atividades/atividade12} que pergunte ao usuário o nome de um arquivo de texto e para cada palavra do arquivo diga quantas vezes ela aparece no texto. \\
      \begin{minted}{bash}
      $ cat arquivo.txt
      a casa que vivo é boa.
      boa casa é.
      $ ./contaPalavras.sh
      Informe o arquivo: arquivo.txt
      casa: 2
      boa:  2
      é:    2
      a:    1
      que:  1
      vivo: 1
      \end{minted}
\end{frame}

\section{Atividade 13}
    \begin{frame}
      \begin{beamercolorbox}[sep=8pt,center,shadow=true,rounded=true]{title}
      \usebeamerfont{title}\insertsectionhead\par%
      \end{beamercolorbox}
   \end{frame}
 
   \begin{frame}
      \frametitle{Exercícios - Parte I}
      Refaça o exercício de sala anterior usando o nome \textit{hosts.sh} na pasta \textit{atividades/atividade13}, porém caso o usuário não forneça nenhuma opção por parâmetros, o \textit{script} deve exibir um menu com as opções disponíveis (adicionar, remover, procurar e listar). Se a opção envolver leitura de valores, deve-se requisitar a entrada adequada do usuário.
   \end{frame}

\section{Atividade 14}
    \begin{frame}
      \begin{beamercolorbox}[sep=8pt,center,shadow=true,rounded=true]{title}
      \usebeamerfont{title}\insertsectionhead\par%
      \end{beamercolorbox}
   \end{frame}
 
   \begin{frame}[fragile]
      \frametitle{Exercícios - Parte I}
      Coloque a resposta em \textit{atividades/atividade14}. Faça um \textit{script} chamado \textit{contaPalavras.awk} que ao ser invocado com o nome de um arquivo de texto diga quantas vezes cada palavra aparece no texto. \\
      \begin{minted}{bash}
      $ cat arquivo.txt
      a casa que vivo é boa.
      boa casa é.
      $ awk -f contaPalavras.awk arquivo.txt 
      Informe o arquivo: arquivo.txt
      casa: 2
      boa:  2
      é:    2
      a:    1
      que:  1
      vivo: 1
      \end{minted}
\end{frame}

\section{Atividade 15}
    \begin{frame}
      \begin{beamercolorbox}[sep=8pt,center,shadow=true,rounded=true]{title}
      \usebeamerfont{title}\insertsectionhead\par%
      \end{beamercolorbox}
   \end{frame}
 
   \begin{frame}
      \frametitle{Exercícios I}
      Coloque os arquivos no pasta \textit{atividades/atividade15}. Vamos fazer um pequeno servidor de arquivos.
      \begin{itemize}
         \item O \textit{script} \textit{cliente.sh} deve receber três parâmetros: IP do Servidor, Porta do Servidor e o nome de um arquivo. Ele deve se conectar ao servidor enviando além do nome do arquivo desejado, o IP e Porta na qual espera o recebimento do arquivo.  
	 \item O \textit{script} \textit{servidor.sh} deve escutar no IP e Porta fornecidos ao cliente. A cada conexão, deve procurar o arquivo requisitado no diretório no qual está executando e enviá-lo através do IP e Porta fornecidos pelo cliente.
     \end{itemize}
     Para funcionar, tanto cliente quanto servidor devem estar na mesma rede.
   \end{frame}

\section{Atividade 16}
    \begin{frame}
      \begin{beamercolorbox}[sep=8pt,center,shadow=true,rounded=true]{title}
      \usebeamerfont{title}\insertsectionhead\par%
      \end{beamercolorbox}
   \end{frame}
 
   \begin{frame}
      \frametitle{Exercícios I}
      Coloque o arquivo \textit{compactador.sh} na pasta \textit{atividades/atividade16}. O objetivo do \textit{script} é definir telas para as seguintes ações:
      \begin{enumerate}
         \item Apresentar uma tela requisitando o \textbf{caminho de um diretório}.
	 \item Formar uma \textbf{lista com os nomes dos arquivos} (sem subdiretórios) do diretório citado. O usuário deve escolher um ou mais arquivos.
	 \item Exibir duas \textbf{opções de compactação}: gzip ou b2zip.
	 \item Questionar o \textbf{nome do arquivo compactado} a ser criado.
	 \item Criar o arquivo compactado com os arquivos selecionados do diretório e exibir uma tela de sucesso com o \textbf{nome final do arquivo}.
      \end{enumerate}
   \end{frame}

\end{document}

