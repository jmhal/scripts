\documentclass{beamer}

\usepackage[portuguese]{babel}
\usepackage[utf8]{inputenc}
\usepackage{minted}

\title{Comandos Avançados}
\author[João Marcelo Uchôa de Alencar]{João Marcelo Uchôa de Alencar}
\institute{Universidade Federal do Ceará - Quixadá}

\begin{document}
   \begin{frame}
      \titlepage
   \end{frame}

   \begin{frame}
      \frametitle{ed - Editor de texto orientado a linhas}
      \begin{itemize}
         \item Uso minimalista. 
         \item Utiliza as mesmas expressões regulares que outros comandos e linguagens.
	 \item Ele é usado para exibir, modificar e manipular arquivos de texto. 
      \end{itemize}
   \end{frame}

\begin{frame}[fragile]
   \frametitle{Indicador de Pesquisa}
   Para abrir um arquivo: 
   \begin{minted}{bash}
   $ ed [nomedoarquivo] 
   \end{minted}
   A única coisa que será impressa é a quantidade de \textit{bytes}. \\
   Como é que uso isso ?!? \\
   Digite \slash, e em seguida uma cadeia de caracteres. \\
   Para imprimir todo o texto: 1,\$ p \\
   Procurar início da linha: \^{}cadeia \\
   Procurar fim da linha: cadeia\$ \\
   Para sair do editor: q \\
\end{frame}

\begin{frame}[fragile]
   \frametitle{\textit{Prompt} de Comando}
   Para tornar as coisas mais fáceis, inicie o ed com o \textit{prompt} de comando: 
   \begin{minted}{bash}
   $ ed -p:
   \end{minted}
   Podemos criar um alias para facilitar: \textbf{\$ alias ed='ed -p:'} \\
   Para adicionar texto a um \textit{buffer}: \textbf{a} \\
   Para voltar ao \textit{prompt} de comando, deixe linha com apenas um '.'. \\
   Para salvar em um arquivo: \textbf{w nomedoarquivo} \\
   Imprimir as linhas numeradas: \textbf{1,\$ n} \\
   Ir para uma determinada linha: \textbf{n}, onde n é o número da linha. \\
   Alterar o conteúdo da linha: \textbf{c} \\
\end{frame}

   \begin{frame}
      \frametitle{Substituição}
      Já deve ter notado que \textbf{1,\$} representa uma faixa de valores. \\
      Nesse caso, é da linha um até o final do arquivo. \\
      Porém poderia ser qualquer faixa: \textbf{1,5 1,10 5,15} etc. \\
      O \textbf{p} é função imprimir, o \textbf{n} numerar as linhas, etc. \\
      O \textbf{g} permite a busca não por cadeias, mas sim expressões regulares: \textbf{g \slash expressao \slash p} \\
      O \textbf{s} é a função substituir: \textbf{1,\$ s\slash expressao \slash cadeia \slash g} \\
      Observer o \textbf{g} no comando anterior. 
   \end{frame}

%   \begin{frame}
%      \frametitle{Exercícios - Parte I}
%      Este exercício não conta como atividade, já que o uso do ed é interativo. Façam para praticar os conhecimentos. 
%      \begin{itemize}
%         \item Crie um arquivo texto chamado meu\_historico.txt com o ed, coloque seu nome na primeira linha e o nome de cada professor que você teve aula nas seguintes.
%	       \item Salve o arquivo, saia do ed, abra novamente e imprima o conteúdo do arquivo.
%	       \item Mude o conteúdo da primeira linha, no lugar do seu nome, coloque Aluno: ou Aluna:, de acordo com sua identidade, seguido pelo seu nome. 
%	       \item Copie o arquivo $/home/compartilhado/hosts$ e utilizando o ed retire todas as linhas que são comentários (começam com \#).
%      \end{itemize}
%   \end{frame}

\begin{frame}[fragile]
   \frametitle{O comando sed}
   \begin{minted}{bash}
   sed expressão regular [arquivo]
   \end{minted}
   A expressão regular é do formato: \\
   $[<$endereço-1$>$, $[<$endereço-2$>]]$ $<$função$>$ $[$argumento$]$
\end{frame}

\begin{frame}[fragile]
   \frametitle{Funções}
   \scriptsize
   Substituir por cadeia:
   \begin{minted}{bash}
   sed 's/expressaoregular/cadeia/' arquivo
   \end{minted}
   Substituir por caractere:
   \begin{minted}{bash}
   sed 'y/caractere-1/caractere-2/' arquivo 
   \end{minted}
   Imprimir linhas que casam com expressão:
   \begin{minted}{bash}
   sed '/expressao/p' arquivo 
   \end{minted}
   Deletar linhas:
   \begin{minted}{bash}
   sed '1,4d' arquivo 
   sed '/expressao/d' arquivo 
   \end{minted}
   Adicionar uma linha em posição:
   \begin{minted}{bash}
   sed '1a \novalinha' arquivo 
   \end{minted}
   A opção -n reduz o que é impresso na tela, enquanto a opção -i atualiza o arquivo. \\
   A opção -r ativa suporte a metacaracteres avançados nas expressões regulares. \\
\end{frame}

\begin{frame}[fragile]
   \frametitle{cut - Cortando cadeias de caracteres}
   \begin{minted}{bash}
   cut -f[campo] -d[delimitador] arquivo 
   \end{minted}
   O \textit{delimitador} indica qual caractere é usado para separar os campos. \\
   O número \text{campo} indica qual campo deve ter a coluna impressa. \\
\end{frame}

\begin{frame}[fragile]
   \frametitle{tr - Transformar cadeias de caracteres}
   \begin{minted}{bash}
tr [opções] <caracteresoriginais> <caracteresdestino>
   \end{minted}
   Converte dos caracteres originais para os caracteres de destino de acordo com a ordem que aparecem. \\
   Pode usar as classes POSIX. \\
   A opção -s elimina caracteres repetidos. \\
\end{frame}

\begin{frame}[fragile]
   \frametitle{uniq - Eliminando repetições}
   \begin{minted}{bash}
uniq arquivo 
   \end{minted}
   Deixa uma única cópia para cada linha repetida. \\
   As linhas devem ser adjacentes. \\
\end{frame}

\begin{frame}[fragile]
   \frametitle{Aritmética no Bash}
   \begin{minted}{bash}
expr <operando1> <operador> <operando2>
   \end{minted}
   O operador pode ser $+,-,*,/,\%$ \\
   É extremamente sensível aos espaços. \\
\end{frame}

\begin{frame}[fragile]
   \frametitle{Aritmética no Bash}
   \begin{minted}{bash}
bc [arquivo]
   \end{minted}
   É como se fosse um \textit{bash} para arimética. \\
   Mais utilizado como comando destino de um pipe. \\
   Aceita o parâmetro \textit{scale}.\\
\end{frame}

\begin{frame}[fragile]
   \frametitle{Aritmética}
   Também é possível usar o próprio Bash para expressões aritméticas inteiras.
   \begin{minted}{bash}
   $(( expressão ))
   \end{minted}
   No lugar de executar um comando, o parênteses extra calcular a expressão. \\
\end{frame}


\begin{frame}[fragile]
   \frametitle{Exercício 03 - Parte II}
   Coloque o \textit{script} na pasta \textit{exercicios/exercicio03}. O nome do \textit{script} deve ser \textit{soma.sh}. Instruções:
   \begin{itemize}
      \item Esse \textit{script} deve considerar que no mesmo diretório que será executado existe um arquivo chamado \textit{compras.txt} com a seguinte informação:
      \begin{minted}{bash}
Playstation5 4999
MacBook 9799
GalaxyS20Ultra 5499
iPad 3499
SmartTV554k 2799
      \end{minted}
      \item Mais informações no próximo \textit{slide}.
   \end{itemize}
\end{frame}


\begin{frame}[fragile]
   \frametitle{Exercício 03 - Parte II - Continuação}
   Você deve fazer uma sequência de cinco comandos, ligados por \textit{pipe}, para somar os valor total
   das compras. Abaixo, a posição do comando no \textit{pipe} e a respectiva funcionalidade que deve 
   executar:
   \begin{enumerate}
      \item Exibe o conteúdo do arquivo.
      \item Recupera apenas os preços (sugestão: \textit{cut}).
      \item Substitui o caractere de nova linha por + (sugestão:  \textit{tr}).
      \item O substitui o + ao final por um caractere nova linha (sugestão: \textit{sed}).
      \item Faz a soma dos valores (sugestão: \textit{bc}).
   \end{enumerate}
   Faça de forma incremental, adicionando um comando por vez no \textit{pipe}.
\end{frame}

%\begin{frame}
%   \frametitle{Exercícios - Parte I}
%   Continuando no diretório \textit{atividades/atividade04}. Copie o arquivo \textit{/home/compartilhado/atividade04.py} para o diretório criado. Crie o arquivo \textit{parte\_01.txt} e coloque os comandos \textit{sed} que realizam as seguintes ações:
%   \begin{enumerate}
%      \item Troque o interpretador \textit{Python} para {/usr/bin/python3}.
%      \item Em um único comando, altere o nome das variáveis \textit{nota1}, \textit{nota2} e \textit{notaFinal} para usarem apenas letras maiúsculas (por exemplo, \textit{nota1} vira \textit{NOTA1}).
%      \item Altere o arquivo para importar o módulo \textit{time} do \textit{Python}.
%      \item Adicione um comando ao final do programa para imprimir a data atual (\textit{time.ctime()}).
%   \end{enumerate}
%\end{frame}

%\begin{frame}
%   \frametitle{Exercícios - Parte II}
%   Ainda no diretório \textit{atividades/atividade04}, crie o arquivo \textit{parte\_02.txt}. Usando os comandos aprendidos, crie uma sequência de comandos ligados por $|$ 
%que realize as ações abaixo. Coloque uma sequência por linha no arquivo \textit{parte\_02.txt}.
%   \begin{enumerate}
%      \item Analisando $/$etc$/$passwd, liste, um por linha, os diferentes shells utilizados pelos usuários (último campo separado por :). 
%      \item Veja o arquivo $/$home$/$compartilhado$/$despesas.txt. Use os comandos para calcular o total de despesas.
%   \end{enumerate}
%\end{frame}

\end{document}

