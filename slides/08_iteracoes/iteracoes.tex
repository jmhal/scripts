\documentclass{beamer}

\usepackage[portuguese]{babel}
\usepackage[utf8]{inputenc}
\usepackage{minted}

\title{Iteração}
\author[João Marcelo Uchôa de Alencar]{João Marcelo Uchôa de Alencar}
\institute{Universidade Federal do Ceará - Quixadá}

\begin{document}
   \begin{frame}
      \titlepage
   \end{frame}

   \begin{frame}[fragile]
      \frametitle{A estrutura do \textit{for}}
      \begin{minted}{bash}
      for var in valor1 valor2 ... valorN
      do
         <comando1>
         <comando2>
         <...>
         <comandoM>
      done
      \end{minted}
      A variável \$\{var\} assume um valor diferente para cada iteração. \\
      Um comando pode ser usado para gerar a lista de valores. Exemplo: \textbf{seq} permite gerar uma sequência de valores inteiros. \\
      \begin{minted}{bash}
      $ seq 10 
      $ seq 1 10 
      $ seq 1 2 10 
      \end{minted}
      A opção $-s$ permite mudar o caractere que separa os números gerados. Se não existir a palavra \textbf{in} e a sequência de valores, \$\{var\} irá percorrer a lista de parâmetros. \\
\end{frame}

   \begin{frame}
      \frametitle{Exemplos}
      \begin{itemize}
         \item Vamos criar um \textit{script} chamado \textbf{somaTotal.sh} que receba um parâmetro e calcule a soma de todos os inteiros de zero até o valor do parâmetro.
	 \item Agora vamos fazer um \textit{script} chamado \textbf{maiorDeTodos.sh} que receba vários parâmetros inteiros e devolva o maior deles.
      \end{itemize}
   \end{frame}

   \begin{frame}[fragile]
      \frametitle{O Separador Entre os Campos - IFS}
      Comandos como o \textbf{sort}, o \textbf{cut} e o próprio \textbf{for} na geração de sua lista de valores usam o espaço em branco como separador padrão. \\
      \begin{minted}{bash}
      usuario=$1
      grep $usuario /etc/passwd
      for info in $(grep $usuario /etc/passwd)
      do
         echo $info
      done
      OLDIFS=$IFS
      IFS=":"
      for info in $(grep $usuario /etc/passwd)
      do
         echo $info
      done
      IFS=$OLDIFS
      \end{minted}
\end{frame}

   \begin{frame}[fragile]
      \frametitle{A estrutura do \textit{while}}
      \begin{minted}{bash}
      while <comando>
      do
          <comando1>
	  <comando2>
	  <...>
	  <comandoN>
      done
      \end{minted}
      O \textbf{comando} é sempre executado no início. Se o resultado for sucesso, então os outros comandos começam a ser executados. Enquanto \textbf{comando} for sucesso, o laço se repete.
\end{frame}

   \begin{frame}[fragile]
      \frametitle{A estrutura do \textit{until}}
      \begin{minted}{bash}
      until <comando>
      do
          <comando1>
	  <comando2>
	  <...>
	  <comandoN>
      done
      \end{minted}
      Só começa a executar os comandos caso o \textbf{comando} da condição seja fracasso. Se após a execução do bloco e \textbf{comando} ainda for fracasso o laço se repete. Só irá terminar quando \textbf{comando} executar sucesso. \\ 
      Para finalizar, só lembrando que tanto o \textbf{break} e o \textbf{continue} das outras linguagens funcionam no \textit{Shell}.
\end{frame}

   \begin{frame}
      \frametitle{Exemplos}
      \begin{itemize}
         \item Faça um \textit{script} que fique em execução sem parar, sendo que cada vez que um usuário entrar ou sair do sistema, ele exiba um aviso de mudança no número de usuários.
	      \item Incremente o \textit{script} anterior. Faça com que a mensagem exibida informa se o usuário entrou ou saiu, e qual o nome do usuário.
	      \item Faça um \textit{script} que receba como parâmetro o caminho de um diretório que só tem arquivos de texto como conteúdo. O \textit{script} deve imprimir em ordem crescente uma lista com os arquivos de acordo com a quantidade de linhas de cada arquivo. Lembrando que o arquivo com mais linhas não é necessariamente o arquivo com tamanho maior em \textit{bytes}.
      \end{itemize}
   \end{frame}

%   \begin{frame}
%      \frametitle{Atividades}
%      \begin{itemize}
%         \item Crie o diretório \textit{atividades/atividade06}
%	 \item Você vai desenvolver o \textit{script} \textit{latencia.sh}
%	 \item Esse \textit{script} vai receber como parâmetro o nome de um arquivo de texto, contendo um endereço IP por linha.
%	 \item O \textit{script} deve usar o comando \textbf{ping} para enviar dez pacotes ICMP para cada endereço do arquivo, calculando o valor médio do tempo de resposta.
%	 \item O \textit{script} deve imprimir uma lista de IP ordenada do menor para o maior tempo médio de resposta, informando além do endereço, o tempo de resposta médio.
%      \end{itemize}
%      No \textbf{próximo slide} tem um exemplo de como utilizar o \textit{script}.
%   \end{frame}
%
%   \begin{frame}[fragile]
%   \frametitle{Atividades}
%   \begin{minted}{bash}
%   # os IPs abaixo são fantasia. 
%   $ cat enderecos_ip.txt
%   8.8.8.8
%   192.168.0.1
%   54.230.57.207
%   $ ./latencia.sh enderecos_ip.txt
%   192.168.0.1 11.1ms
%   54.230.57.207 55.4ms
%   8.8.8.8 94.0ms
%   \end{minted}
%\end{frame}

\end{document}
