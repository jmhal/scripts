\documentclass{beamer}

\usepackage[portuguese]{babel}
\usepackage[utf8]{inputenc}
\usepackage{minted}

\title{Scripts de Inicialização}
\author[João Marcelo Uchôa de Alencar]{João Marcelo Uchôa de Alencar}
\institute{Universidade Federal do Ceará - Quixadá}


\begin{document}
   \begin{frame}
      \titlepage
   \end{frame}

   \begin{frame}
      \frametitle{Execução de Scripts de Inicialização}
      \begin{itemize}
          \item A inicialização de um sistema envolve a execução de \textit{scripts}.
          \item O sistema clássico de inicialização era o modelo System V:
	  \begin{itemize}
	     \item \textit{Scripts} comuns em \textit{shell}.
	     \item Organizados nas pastas \textit{/etc/init.d} ou \textit{/etc/rc.d}.
	     \item Cabeçalho especial para determinar a ordem de execução.
          \end{itemize}
          \item As distribuições ainda suportam System V de forma legada.  
      \end{itemize}
   \end{frame}

   \begin{frame}[fragile]
      \frametitle{Exemplo de um Script System V}
      \begin{minted}{bash}
#! /bin/sh

### BEGIN INIT INFO
# Provides:             sshd
# Required-Start:       $remote_fs $syslog
# Required-Stop:        $remote_fs $syslog
# Default-Start:        2 3 4 5
# Default-Stop:
# Short-Description:    OpenBSD Secure Shell server
### END INIT INFO

set -e

# /etc/init.d/ssh: start and stop the OpenBSD "secure shell(tm)" daemon

test -x /usr/sbin/sshd || exit 0
( /usr/sbin/sshd -\? 2>&1 | grep -q OpenSSH ) 2>/dev/null || exit 0

umask 022
...
      \end{minted}
\end{frame}

   \begin{frame}
      \frametitle{Introdução ao SystemD}
      \begin{itemize}
         \item O System V foi substituído pelo SystemD.
	 \item Assume o controle logo após a ativação do \textit{kernel} pelo GRUB.
         \item Integrou várias funcionalidades extras em relação ao System V.
	 \item Exige arquivos de configuração que não são \textit{shell scripts}.
         \item Utiliza unidades de \textit{sockets} para coordenar a inicialização simultânea de vários serviços.
      \end{itemize}
      Sua adoção causou polêmica no início, entretanto hoje é utilizado por quase todas distribuições.
   \end{frame}

   \begin{frame}
      \frametitle{Arquitetura do SystemD}
      As tarefas de inicialização são organizadas em \textbf{unidades}.
      \begin{itemize}
         \item \textbf{service}: são tarefas para inicializar serviços como servidores \textit{web}, bancos de dados, etc.	      
         \item \textbf{timer}: temporizadores para determinar ações com base no tempo.
         \item \textbf{mount}: arquivo de configuração para controlar a montagem de um volume.
         \item \textbf{target}: grupos de unidades que compõem um serviço complexo.
      \end{itemize}
   \end{frame}

   \begin{frame}
      \frametitle{Arquitetura do SystemD}
      \begin{itemize}
         \item \textbf{snapshot}: cria estados do sistema para restauração.	      
         \item \textbf{path}: tarefas para monitorar diretórios.
	 \item \textbf{socket}: arquivo de configuração sobre \textit{sockets} de rede ou arquivo FIFO.
	 \item \textbf{swap}: gerencia partições de \textit{swap}.
      \end{itemize}
   \end{frame}

   \begin{frame}[fragile]
      \frametitle{Comandos do SystemD}
      \begin{minted}{bash}
# Listar todos os serviços
systemctl list-unit-files --type=service
# Verificar se um serviço está ativo
systemctl is-active httpd.service
# Habilitá-lo no próximo boot
systemctl enable httpd.service
# Desabilitá-lo no próximo boot
systemctl disable httpd.service
      \end{minted}
\end{frame}

   \begin{frame}[fragile]
      \frametitle{Comandos do SystemD}
      \begin{minted}{bash}
# Iniciar de imediato um serviço
systemctl start httpd.service
# Reiniciar de imediato um serviço
systemctl restart httpd.service
# Parar um serviço
systemctl stop httpd.service
# Forçar um serviço a recarregar sua configuração
systemctl reload httpd.service
# Verificar a situação de um serviço
systemctl status httpd.service
      \end{minted}
\end{frame}

   \begin{frame}
      \frametitle{Criando um Serviço SystemD}
      \begin{itemize}
          \item Vamos criar um serviço em \textit{shell script} 
	  \item Trata-se de \textit{script} executado durante a inicialização.
          \item O \textit{script} vai finalizar, mas poderia ficar em execução enquanto a máquina estiver ativa.
          \item Vamos fazer em \textit{shell}, mas poderia ser feito em qualquer outra linguagem.
      \end{itemize}
      Nosso \textit{script} vai verificar o tamanho do diretório \textit{/home} e escrever um relatório no diretório \textit{/root}.
   \end{frame}

   \begin{frame}[fragile]
      \frametitle{Unidade do Serviço}
      Arquivo \textit{disk-space-check.service}:
      \begin{minted}{bash}
[Unit]
After=network.target

[Service]
ExecStart=/usr/local/bin/disk-space-check.sh

[Install]
WantedBy=default.target
      \end{minted}
      Deve ser colocado no diretório \textit{/etc/systemd/system}.
\end{frame}

   \begin{frame}
      \frametitle{Unidade do Serviço}
      \begin{itemize}
         \item \textbf{After}: quando o \textit{script} pode iniciar. Aqui estamos dizendo que ele deve executar após a rede estar configurada, apenas por exemplo.	      
         \item \textbf{ExecStart}: caminho completo do executável.
	 \item \textbf{WantedBy}: o \textit{target} no qual o serviço deve ser executado.
      \end{itemize}
      Existem muito mais detalhes, mas vai ficar no básico nesta introdução.
   \end{frame}

   \begin{frame}
   \end{frame}  

   \begin{frame}[fragile]
      \frametitle{Usando o SystemD}
      Além do Ubuntu, várias outras distribuição estão adotando
      \scriptsize
      \begin{minted}{bash}
      $ sudo cp disk-space-check.service /etc/systemd/system/
      $ sudo chmod 744 disk-space-check.sh
      $ sudo chmod 664 /etc/systemd/system/disk-space-check.service
      $ sudo systemctl daemon-reload
      $ sudo systemctl enable disk-space-check.service
      $ sudo systemctl status disk-space-check.service 
      \end{minted}
\end{frame}

    \begin{frame}
	\frametitle{Referências}
	\begin{enumerate}
	   \item \url{https://linuxconfig.org/how-to-automatically-execute-shell-script-at-startup-boot-on-systemd-linux}
           \item \url{https://medium.com/@benmorel/creating-a-linux-service-with-systemd-611b5c8b91d6}
           \item \url{https://e-tinet.com/linux/systemd/}
	\end{enumerate}
    \end{frame}

\end{document}
