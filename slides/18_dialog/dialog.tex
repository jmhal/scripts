\documentclass{beamer}

\usepackage[portuguese]{babel}
\usepackage[utf8]{inputenc}
\usepackage{minted}
\setbeamertemplate{footline}[frame number]

\title{Dialog e Derivados}
\author[João Marcelo Uchôa de Alencar]{João Marcelo Uchôa de Alencar}
\institute{Universidade Federal do Ceará - Quixadá}

\begin{document}
   \begin{frame}
      \titlepage
   \end{frame}

   \begin{frame}
      \frametitle{Introdução}
      \begin{itemize}
         \item \textit{Dialog} é um programa usado para desenhar interfaces amigáveis para o usuário. 
	 \item Ele é um comando como qualquer outro, podendo ser invocado diretamente no terminal ou inserido em \textit{scripts}.
	 \item Existem alguns derivados que criam telas em servidores X e ambientes como Gnome e KDE.
	 \item \textit{sudo apt install dialog}
	 \item Elementos
	 \begin{itemize}
	    \item Botões.
	    \item Janelas.
	    \item Menu.
	    \item Entradas para texto.
	 \end{itemize}
      \end{itemize}
   \end{frame}

   \begin{frame}[fragile]
      \frametitle{Meu Primeiro Exemplo em Dialog}
      \begin{minted}{bash}
$ dialog --msgbox "Programação de Scripts" 5 40
      \end{minted}
      \begin{itemize}
         \item Desenhamos uma caixa de mensagens (\textit{msgbox}).
	 \item O tamanho dela é 5 por 40.
	 \item São linhas e colunas, nada de \textit{pixels}.
      \end{itemize}
\end{frame}

   \begin{frame}
      \frametitle{As 15 Caixas Mágicas}
      \scriptsize
      \begin{table}
         \begin{tabular}{c|c}
         \textbf{Tipo} & \textbf{Ação} \\
	 \hline
	 calendar & Calendário \\
         \hline 
 	 checklist & Lista de opções \\
         \hline 
 	 fselect & Escolha de arquivo \\
         \hline 
 	 gauge & Barra de progresso \\
         \hline 
 	 infobox & Mensagem \\
         \hline 
 	 inputbox & Entrada de Texto \\
         \hline 
 	 menu & Menu de opções \\
         \hline 
 	 msgbox & Mensagem com botão \\
         \hline 
 	 passwordbox & Digita uma senha \\
         \hline 
 	 radiolist & Lista de opções \\
         \hline 
 	 tailbox & Saída do comando \textit{tail -f} \\
         \hline 
 	 tailboxbg & Saída no \textit{background} \\
         \hline 
 	 textbox & Conteúdo de um arquivo \\
         \hline 
 	 timebox & Escolhe um horário \\
         \hline 
 	 yesno & Pergunta Sim/Não \\
         \hline 
       \end{tabular}
      \end{table}
   \end{frame}


   \begin{frame}[fragile]
      \frametitle{Exemplos de Comandos}
      \scriptsize
      \begin{minted}{bash}
$ dialog --title "Escolha a data" --calendar "" 0 0 21 12 1999
      
$ dialog --title "Escolha onde instalar" --fselect /usr/share/vim 0 0 

$ dialog --title "Instalação dos Pacotes" \ 
--gauge "\nInstalando vim-6.0a.tgz..." 8 40 60

$ dialog --title "Aguarde" --infobox "\nFinalizando em 5 segundos..." 0 0 

$ dialog --title "Confirmação" --passwordbox "Confirme a senha:" 0 0 

$ dialog --title "Perfil" --menu "Escolha o perfil de instalação:" \ 
0 0 0 mínima "Instala o mínimo" completa "Instala tudo" \ 
customizada "Você escolhe"
      \end{minted}
\end{frame}

   \begin{frame}[fragile]
      \frametitle{Exemplos de Comandos}
      \scriptsize
      \begin{minted}{bash}
$ dialog --title "Parabéns" --msgbox \ 
"Instalação finalizada com sucesso." 6 40
      
$ dialog --title "Pergunta" --radiolist "Há quanto tempo você usa o Vi?" \
0 0 0 iniciante "até um ano" on experiente \
"mais de um ano" off guru "mais de três anos" off 

$ tail -f /var/log/syslog > out & dialog \
--title "Monitorando mensagens do Sistema" --tailbox out 0 0 

$ dialog --title "Visualizando Arquivo" \
--textbox "/usr/share/vim/vim74/indent.vim" 0 0  

$ dialog --title "Ajuste o Relógio" \ 
--timebox "\nDica: use as setas e o TAB." 0 0 23 59 30 

$ dialog --title "AVISO" --yesno \
"\nO Vi foi instalado e configurado. Executá-lo?" 0 0 
      \end{minted}
\end{frame}

   \begin{frame}
      \frametitle{Agora, como recuperar a entrada?}
      \begin{itemize}
         \item A linha de comando é sempre longa, com várias opções.
	 \item O texto é redimensionado na caixa de maneira automática.
	 \item O \textbf{código de retorno} permite saber se o usuário escolheu \textbf{Sim/Não, Ok/Cancelar}.
	 \item A \textbf{saída de erro, STDERR,} permite recuperar os textos e itens escolhidos.
      \end{itemize}
   \end{frame}

   \begin{frame}[fragile]
      \frametitle{Parâmetros Obrigatórios da Linha de Comando}
      \begin{minted}{bash}
$ dialog --tipo-da-caixa '<texto>' <altura> <largura> 
      \end{minted}
      \begin{itemize}
         \item texto: palavra ou frase que aparece no ínicio da caixa.
	 \item altura: número de linhas da caixa.
	 \item largura: número de colunas da caixa.
      \end{itemize}
\end{frame}

   \begin{frame}[fragile]
      \frametitle{Como reconhecer respostas SIM ou NÃO}
      \begin{minted}{bash}
$ dialog --yesno "sim ou não?" 0 0 
      \end{minted}
      Como saber o que foi escolhido?
      \begin{minted}{bash}
$ dialog --yesno "sim ou não?" 0 0; echo Retorno: $?; 
      \end{minted}
      \begin{block}{Memorizando}
      SIM=0, NÃO=1
      \end{block}
      \begin{minted}{bash}
$ dialog --yesno "Quer ver as horas?" 0 0 \
&& echo "Agora são: $(date)"; 
      \end{minted}
\end{frame}

   \begin{frame}[fragile]
      \frametitle{Exemplo Sim/Não}
      \scriptsize
      \begin{minted}{bash}
#!/bin/sh
# lsj.sh -- o script do "ls joiado"
# Este script faz parte do http://aurelio.net/shell/dialog

# Zerando as opções
cor= ; ocultos= ; subdir= ; detalhes=

# Obtendo as configurações que o usuário deseja
dialog --yesno 'Usar cores?'               0 0 && cor='--color=yes'
dialog --yesno 'Mostrar arquivos ocultos?' 0 0 &&  ocultos='-a'
dialog --yesno 'Incluir sub-diretórios?'   0 0 &&   subdir='-R'
dialog --yesno 'Mostrar visão detalhada?'  0 0 && detalhes='-l'

# Mostrando os arquivos
ls $cor $ocultos $subdir $detalhes      
      \end{minted}
\end{frame}

   \begin{frame}[fragile]
      \frametitle{Como obter o texto que o usuário digitou?}
      \begin{minted}{bash}
$ dialog --inputbox "Digite seu nome:" 0 0 
      \end{minted}
      Como guardar o que foi usado em uma variável? \\
      \begin{itemize}
         \item Redirecionar a saída de erros (stderr) para um arquivo e depois ler o conteúdo do mesmo.
	 \item Redirecionar a saída de erros (stderr) para a saída padrão (stdout).
	 \item Usar a opção $--stdout$ do dialog.
      \end{itemize}
      \begin{minted}{bash}
$ nome=$(dialog --stdout --inputbox \
"Digite seu nome:" 0 0)
$ echo $nome
      \end{minted}
\end{frame}

   \begin{frame}[fragile]
      \frametitle{Como obter o item de um menu?}
      \begin{minted}{bash}
$ dialog --menu "Opções:" 0 0 0 1 um 2 dois 3 três
      \end{minted}
      Como guardar a opção em uma variável? \\
      \begin{minted}{bash}
$ opcao=$(dialog --stdout --menu "Opções:" \ 
0 0 0 1 um 2 dois 3 três)
$ echo $opcao
$ opcao=$(dialog --radiolist "As cores: " 0 0 0 \
   amarelo "a cor do sol" OFF \
   verde "a cor da grama" ON \
   azul "a cor do céu" OFF)
$ echo $opcao
      \end{minted}
\end{frame}

   \begin{frame}[fragile]
      \frametitle{Como obter múltiplos itens de uma lista?}
      \begin{minted}{bash}
estilos=$( dialog --stdout \
	--checklist 'Você gosta de:' 0 0 0 \
	rock  '' ON  \
	samba '' OFF \
	metal '' ON  \
	jazz  '' OFF \
	pop   '' ON  \
	mpb   '' OFF )
echo "Você escolheu: $estilos"      
      \end{minted}
\end{frame}

   \begin{frame}[fragile]
      \frametitle{Como obter múltiplos itens de uma lista?}
      \begin{minted}{bash}
estilos=$( dialog --stdout \
	--separate-output \
	--checklist 'Você gosta de:' 0 0 0 \
	rock  '' ON  \
	samba '' OFF \
	metal '' ON  \
	jazz  '' OFF \
	pop   '' ON  \
	mpb   '' OFF )

echo "$estilos" | while read LINHA
do
 	echo "--- $LINHA"
done      
      \end{minted}
\end{frame}

   \begin{frame}[fragile]
      \frametitle{Botão Cancelar/Esc/Help}
      \footnotesize
      \begin{minted}{bash}
case $? in
	  0) echo O usuário apertou o botão OK (ou o Yes) ;;
	  1) echo O usuário apertou o botão CANCELAR (ou o No) ;;
	  2) echo O usuário apertou o botão HELP ;;
	255) echo O usuário apertou a tecla ESC ;;
	  *) echo Retorno desconhecido;;
esac
      \end{minted}
\end{frame}

   \begin{frame}
      \frametitle{Exemplo}
      Podemos fazer uma agenda de nomes/e-mails com \textit{dialog}?
      \scriptsize
      \begin{enumerate}
         \item Exibir menu com opções: Adicionar, Remover, Exibir Usuário, Exibir Todos e Sair.
	 \item Adicionar
	 \begin{enumerate}
            \scriptsize
	    \item Exibir diálogo para receber nome.
	    \item Exibir diálogo para receber e-mail.
	    \item Adicionar no arquivo.
	 \end{enumerate}
	 \item Remover
	 \begin{enumerate}
            \scriptsize
	    \item Exibir diálogo para receber e-mail.
	    \item Remover todos os nomes com o mesmo e-mail.
	 \end{enumerate}
	 \item Exibir Usuário
	 \begin{enumerate}
            \scriptsize
	    \item Exibir diálogo para receber e-mail.
	    \item Exibir o nome e o e-mail, caso contrário afirmar que usuário não existe.
	 \end{enumerate}
	 \item Exibir Todos
	 \begin{enumerate}
            \scriptsize
	    \item Exibir todos os e-mails cadastrados.
	 \end{enumerate}
	 \item Sair
      \end{enumerate}
      \normalsize
      Após executar as opções 2, 3, 4 ou 5, deve retornar ao menu principal.
   \end{frame}

   \begin{frame}
      \frametitle{Atividades}
      Coloque o arquivo \textit{compactador.sh} na pasta \textit{atividades/atividade11}. O objetivo do \textit{script} é definir telas para as seguintes ações:
      \begin{enumerate}
         \item Apresentar uma tela requisitando o \textbf{caminho de um diretório}.
	 \item Formar uma \textbf{lista com os nomes dos arquivos} (sem subdiretórios) do diretório citado. O usuário deve escolher um ou mais arquivos.
	 \item Exibir duas \textbf{opções de compactação}: gzip ou b2zip.
	 \item Questionar o \textbf{nome do arquivo compactado} a ser criado.
	 \item Criar o arquivo compactado com os arquivos selecionados do diretório e exibir uma tela de sucesso com o \textbf{nome final do arquivo}.
      \end{enumerate}
   \end{frame}

\end{document}
